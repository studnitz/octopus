\documentclass[accentcolor=tud0b,12pt,paper=a4]{tudreport}

\usepackage[utf8]{inputenc}
\usepackage{ngerman}
\usepackage{parcolumns}

\newcommand{\titlerow}[2]{
	\begin{parcolumns}[colwidths={1=.15\linewidth}]{2}
		\colchunk[1]{#1:} 
		\colchunk[2]{#2}
	\end{parcolumns}
	\vspace{0.2cm}
}

\title{Vernetztes Video-Capture-Tool}
\subtitle{Qualitätssicherungsdokument}
\subsubtitle{%
	\titlerow{Gruppe 29}{%
		Yannick Michael Schädle <yannick\_michael.schaedle@stud.tu-darmstadt.de>\\
		Jobst Alexander Carl von Studnitz <jobst\_alexander\_carl.von\_studnitz@stud.tu-darmstadt.de>\\
		Nicolas Paul Schickert <nicolas\_paul.schickert@stud.tu-darmstadt.de>\\
		Bartosz Pawel Milejski <bartosz.milejski@stud.tu-darmstadt.de>}
	\titlerow{Teamleiter}{Nicole Wagner <nicole.wagn@googlemail.com>}
	\titlerow{Auftraggeber}{%
		Dipl.-Ing. Torsten Wagner <wagner@iad.tu-darmstadt.de>\\
		M.Sc. Katharina Rönick <k.roenick@iad.tu-darmstadt.de>\\
		Institut für Arbeitswissenschaft\\
		Fachbereich Maschinenbau}
	\titlerow{Abgabedatum}{xx.xx.xxxx}}
\institution{Bachelor-Praktikum WS~15/16\\Fachbereich Informatik}

\begin{document}

	\maketitle
	\tableofcontents 
	
	\chapter{Einleitung}
		Kurze Projektbeschreibung
	
	\chapter{Qualitätsziele}
        \section{Ziel 1}
    
		Beschreibung eines QS-Ziels und wieso dieses in diesem Projekt wichtig ist.\\
		Maßnahme zum Erreichen des Ziels.\\
		Prozessbeschreibung - wer führt wann die oben genannte Maßnahme durch und wie wird auf Probleme reagiert.\\
		Weitere Anforderungen sind den Unterlagen und der Vorlesung zur Projektbegleitung zu entnehmen.

        \section{Ziel xyz}
	        \ldots
	        
	
\appendix	
	\chapter{Anhang}
		(Am Ende des Projekts nachzureichen)\\
		Beleg für durchgeführte Maßnahmen, bzw. falls nicht durchgeführt eine Begründung wieso die Durchführung nicht möglich oder nicht erfolgt ist. \\
		Weitere Anforderungen sind den Unterlagen und der Vorlesung zur Projektbegleitung zu entnehmen.
	
\end{document}

