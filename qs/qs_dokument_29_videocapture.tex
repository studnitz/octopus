\documentclass[accentcolor=tud0b,12pt,paper=a4]{tudreport}

\usepackage[utf8]{inputenc}
\usepackage{ngerman}
\usepackage{parcolumns}

\newcommand{\titlerow}[2]{
	\begin{parcolumns}[colwidths={1=.15\linewidth}]{2}
		\colchunk[1]{#1:} 
		\colchunk[2]{#2}
	\end{parcolumns}
	\vspace{0.2cm}
}

\title{Vernetztes Video-Capture-Tool}
\subtitle{Qualitätssicherungsdokument}
\subsubtitle{%
	\titlerow{Gruppe 29}{%
		Yannick Michael Schädle <yannick\_michael.schaedle@stud.tu-darmstadt.de>\\
		Jobst Alexander Carl von Studnitz <jobst\_alexander\_carl.von\_studnitz@stud.tu-darmstadt.de>\\
		Nicolas Paul Schickert <nicolas\_paul.schickert@stud.tu-darmstadt.de>\\
		Bartosz Pawel Milejski <bartosz.milejski@stud.tu-darmstadt.de>}
	\titlerow{Teamleiter}{Nicole Wagner <nicole.wagn@googlemail.com>}
	\titlerow{Auftraggeber}{%
		Dipl.-Ing. Torsten Wagner <wagner@iad.tu-darmstadt.de>\\
		M.Sc. Katharina Rönick <k.roenick@iad.tu-darmstadt.de>\\
		Institut für Arbeitswissenschaft\\
		Fachbereich Maschinenbau}
	\titlerow{Abgabedatum}{08.11.2015}}
\institution{Bachelor-Praktikum WS~15/16\\Fachbereich Informatik}

\begin{document}

	\maketitle
	\tableofcontents 
	
	\chapter{Einleitung}
    Der Auftraggeber benötigt ein vernetztes Video-Capture-Tool. An zahlreiche Kleinrechner sollen Webcams (später auch Microsoft Kinects und Fotokameras) angeschlossen werden. Diese Webcams zeichnen Bilder auf, die komprimiert und übertragen werden sollen an einen zentralen Server. Falls mit der gegebenen Hardware möglich, soll die Übertragung der Datenströme in Echtzeit geschehen. Auf dem Server läuft eine Applikation, die die Bildströme nebeneinander anzeigt. Die Ausrichtung der einzelnen Bildströme ist veränderbar. Die Aufzeichnungen auf den einzelnen Mini-PCs sollen entweder synchron starten oder von der Applikation so angezeigt werden, als seien sie synchron gestartet worden sein. Das System soll sich ohne zusätzliche Aktionen komplett betriebsbereit machen, sobald die Hardware angeschaltet wird. Als Erweiterung ist eine Steuerung der Anwendung über eine App gewünscht. Unsere Aufgabe ist die Konfiguration der Hard- und Middleware, Planung und Entwicklung der Server Applikation und Erweiterung der Android App.

    Es sollen dabei zwei Dokumentationen angelegt werden. Die Erste ist eine an Entwickler gerichtete Beschreibung unseres Vorgehens bei der Planung, Organisation und Durchführung des Projektes, die zukünftigen Teams als Referenz dienen kann. Die Zweite ist eine für Maschinenbauer verständliche Anleitung zur Einrichtung und Verwendung des Systems.

	
	\chapter{Qualitätsziele}
        \section{Robustheit}
          Im Rahmen dieses Projektes hat die Sicherstellung der Robustheit des vernetzten Video-Capture-Tools höchste Priorität, da garantiert werden muss, dass die Anwendung die geforderten Videos unter flexiblen Bedingungen aufnimmt. So soll es möglich sein, dass unterschiedlich viele Kameras an die Clients angeschlossen werden und auch die Anzahl der Clients, die mit dem Server verbunden sind, problemlos zu ändern. Außerdem soll die Applikation fehlertolerant sein. Sofern eine Kamera während der Aufnahme ausfällt, so sollen die Aufnahmen anderer Kameras davon nicht betroffen sein.

          Um das Qualitätsziel zu erreichen, führen wir in jeder Iteration Code Reviews durch. Auch das Pair Programming ist eine Technik, die uns dabei helfen wird. Dadurch stellen wir sicher, dass bereits bei der Codeentwicklung die Anzahl potentieller Fehler eingeschränkt wird indem Sonderfälle, die nähere Beachtung benötigen, früher erkannt und dementsprechend auch berücksichtigt werden können.

          Zusätzlich zu den Code Reviews werden wir umfangreiche Testpläne erstellen, die soweit es möglich ist, auch automatisiert ablaufen sollen. Durch kontinuierliche Integration werden diese automatischen Tests immer ausgeführt, sobald wir Code in unsere Versionsverwaltung einchecken. Sollte ein Build Fehler enthalten, werden wir vom CI-Tool benachrichtigt und ein Entwickler kann gewählt werden um die Fehler bis zum nächsten Build zu beseitigen.

          Tests, die nicht bzw. nur schwer automatisiert ausführbar sind, wie das Testen von bestimmten Hardware Konfigurationen beim Zusammenspiel von Clients und Kameras, werden in einem manuellen Testplan festgehalten. Dieser Testplan beschreibt detailliert, wie das Tool manuell mit den verschiedenen Hardware Konfigurationen getestet werden muss, was erwartet wird und ab wann ein Test als bestanden gilt.


        \section{Zuverlässigkeit}
	        \ldots
	      \section{Bedienbarkeit}
	        \ldots

	        
	
\appendix	
	\chapter{Anhang}

	
\end{document}
